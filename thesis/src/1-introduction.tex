As the power of computer grows and capability of sensors increases,
more and more incredible things are becoming possible.
One such thing is \gls{har},
for which the applications are many.

Not only is it possible to detect when people are running or walking \cite{running-walking},
but even respiratory rates \cite{breathing}, falling down \cite{falling}, shoplifting \cite{shoplifting},
operating room procedures \cite{operatingroom}, hand gestures \cite{handgestures},
and certain illnesses \cite{parkinsson} have been proven to be recognizable by sensors.
The possibilities seem almost limitless.

For long, the focus in \gls{har} research has been in visible spectrum video.
While it is undeniably a powerful sensing mechanism for the purpose,
it has multiple major downsides.
Visible spectrum imaging is very prone to occlusion and lighting conditions.
In addition, due to face recognition, there are genuine concerns about the privacy issues related to visible spectrum sensing. \cite{sensing-survey}
Many actions can also look very similar depending on the viewing angle.
With additional sensors, extra information can be used to better distinguish otherwise similar action from each other.

Wearable sensors are a well-established mechanism human activity recognition \cite{wearables}.
This is already being used in multiple commercial applications,
such as exercise and sleep recognition in smart watches.
The downside of wearable sensors is that they can be cumbersome and unfashionable
require willful equipping and rarely can be connected to mains.
Clearly, if high-performance remote sensing can be achieved with a reasonable price,
it is the superior alternative.

In addition to visible spectrum imaging,
remote sensors include depth imaging (stereo camera),
infrared imaging, acoustic sensing (microphones) and electromagnetic sensing (e.g. radar).
Different kinds of proximity sensors, such as magnetic switches, pressure sensors, temperature sensors,
and electrostatic proximity sensors can also be used,
although their installation may be more labour-intensive \cite{sensing-survey}.

It has been shown that depth imaging can achieve at least similar performance in activity recognition as visible spectrum imaging,
while simultaneously preserving privacy \cite{depth}.
In complex scenes, using depth imaging can increase the performance of activity recognition substantially \cite{depth-2}.
Depth imaging still suffers from many of the same problems as visible spectrum imaging,
most importantly obstruction.

Radar devices are capable of sensing through visual obstructions
and have been demonstrated to achieve very good performance at \gls{har},
especially when operating on the \gls{mmwave} spectrum.
Another upside for radar imaging is that unlike cameras, 
they are not susceptible to lightning conditions. \cite{radar-survey}
As a downside, they are active devices in the sense that they must also have an active transmitter.
Although passive radars exist, the sensing performance is compromised.
WiFi signals and channel state information have also been demonstrated to be an effective alternative to radar sensing
in environments where WiFi is available \cite{sensing-survey}.

Even with as low as $8 \times 8$ pixel resolution infrared cameras,
recognizing some activities has been proven possible.
In some cases the recognized activities have been very simple,
such as sitting, standing, or lying on ground \cite{ir-simple-activities}.
With some additional context information,
more complex household activities can also be recognized \cite{ir-household}.

Sometimes it bay be hard to distinguish activities from each other based on only a single sensor.
Consider for example cutting vegetables in the kitchen and making a sandwich.
A low resolution infrared sensor might only recognize that a person is occupying the same place,
but may not detect what exactly is being done. A microphone could detect the sound being emitted
or a radar could detect the distinct velocity spectrum of the activity.

Sensor fusion can provably be leveraged to improve \gls{har} accuracy \cite{fusion1, fusion2}.
Not a wide range of data sets exist for this purpose and they often consider only a limited number of sensors \cite{sensing-survey}.
It is important to have a wide range of data sets available for training machine learning models
and to assess the performance of different sensors and algorithms.
In the field of machine learning and \gls{har}, transfer learning has also been a subject of much interest in the recent years \cite{transfer}.

For the aforementioned reason,
it was considered valuable to create a portable multi-modal sensing system that can be used for recording data sets.
As the product of this thesis, such a system was created.
The sensors installed in the systems were a combined visible and depth spectrum camera,
an $8 \times 8$ pixel infrared camera, a $4 \times 4$ channel microphone and a 60 GHz radar.

In Chapter \ref{ch:2-premise}, the sensors used in the system will be presented in more detail.
Additionally, requirements will be out for the developed system.
In Chapter \ref{ch:3-system}, the implementation of the system will be detailed on an architectural level.
Details of the source code will not be discussed. The source code of the project is publicly available for viewing on GitHub \cite{github-link}.
The data formats produced by the recording system will be documented in Chapter \ref{ch:4-files-and-post} along some data processing examples.
The quality and performance of the system will be briefly discussed in Chapter \ref{ch:5-evaluation},
and finally, Chapter \ref{ch:6-conclusion} will conclude the thesis.