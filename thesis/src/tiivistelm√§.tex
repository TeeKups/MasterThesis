Tietokoneiden laskentatehon kasvaminen ja sensorien kyvykkyyden kasvaminen avaa uusia mahdollisuuksia aktiviteetin tunnistukselle.
Aktiviteetintunnistukselle on esitetty useita mahdollisia käyttökohteita kuten terveyspalvelut ja automaattinen myymälävartiointi.
Kaupallisiakin sovelluksia on jo useita, yleisimpänä matkapuhelimien ja älykellojen terveyssovellukset.

Tyypillisesti aktiviteetintunnistus perustuu joko videokuvaan tai puettaviin sensoreihin.
Videokuva kuitenkin on herkkä olosuhteille, kuten valolle ja esteille,
kun taas puettavat sensorit saattavat olla epämukavia, epämuodikkaita, tai muusta syystä epäkäytännöllisiä.

Selvästi vaihtoehtoisille aktiviteetintunnistusmekanismeille on kysyntää.
Lisäksi sensorifuusio on viime vuosina kasvattanut suosiotaan.
Sensorit kykenevät havaitsemaan vain tiettyjä modaliteetteja,
joten usean modaliteetin yhdistämisellä saatetaan parantaa aktiviteetin tunnistustarkkuutta.

Datasettejä, jotka huomioivat sensorifuusion, on kuitenkin vain vähän.
Tästä syystä katsottiin hyödylliseksi kehittää nauhoitusjärjestelmä,
joka yhdistää useita sensoreita.
Järjestelmään valitut sensorit olivat RGB-D kamera, 60 GHz tutka, matalaresoluutioinen infrapunakamera, sekä mikrofoni.
Nauhoitusjärjestelmä toteutettiin käyttäen rinnakkaista ohjelmointimallia datan läpijuoksun maksimoimiseksi.

Kyseinen järjestelmä kehitettiin tämän diplomityön produktina ja tämä diplomityö dokumentoi käytetyt sensorit,
toteutetun järjestelmän arkkitehtuurin, sekä tuotetut dataformaatit.
Toteutettu järjestelmä jäi hieman epävakaaksi,
mutta kuitenkin sitä voidaan sellaisenaan käyttää pienimuotoisten datasettien nauhoittamiseen.
Todennäköisesti epävakaus johtui joko käytetystä tutkalaitteesta tai toteutetun ohjelmiston viasta.
Jos epävakaus saadaan korjattua,
järjestelmä on kykenevä myös laajamuotoisempien datasettien nauhoittamiseen.
Joka tapauksessa, toteutettu järjestelmä toimii hyvänä pohjana tulevalle tutkimukselle ja kehitykselle.