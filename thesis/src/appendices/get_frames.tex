The Listing \ref{lst:get_frames} presents an example for parsing the radar data cubes from the raw radar data.
The meanings of the variables used in the Listing,
that correspond to the symbols in \ref{sec:radar-file},
are defined in \ref{tab:get_frames_variable_definitions}.

\begin{lstlisting}[language=Python, caption={Code example for extracting radar cubes from the raw radar data.}\label{lst:get_frames}]
import numpy

def get_frames(N, M, K, path):
    with open(path, 'rb') as _file:
        data = _file.read()
        
    d = numpy.frombuffer(data, dtype='int16')

    n = numpy.arange(0, len(d), 4)
    d_r = numpy.array([d[n], d[n+1]]).flatten('F')
    d_i = numpy.array([d[n+2], d[n+3]]).flatten('F')
    s = d_r + 1j*d_i
    
    S = s.reshape((-1, M, K, N), order='C') \
            .transpose(0, 2, 1, 3)

    return S
\end{lstlisting}

\begin{table}[]
    \centering
    \begin{tabular}{l l}
        \toprule
            \textbf{Variable} & \textbf{Definition} \\
        \midrule
            \texttt{N} & The number of samples per chirp \\
            \texttt{M} & The number of chirps per sample \\
            \texttt{K} & The number of active receivers \\
            \texttt{d} & Raw 16-byte integer samples ($\vec{d}$) \\
            \texttt{d\_r} & Vector of real samples (in-phase component) \\
            \texttt{d\_i} & Vector of imaginary samples (quadrature component) \\
            \texttt{s} & Vector of complex samples \\
            \texttt{S} & The tensor $\vec{S}$, that contains the radar cubes (frames) $\left[ \vec{S}_0, \ldots, \vec{S}_{ \frac{\|\vec{s}\|}{NMK}} \right]$ \\
        \bottomrule
    \end{tabular}
    \caption{Variable meanings in Listing \ref{lst:get_frames}}
    \label{tab:get_frames_variable_definitions}
\end{table}