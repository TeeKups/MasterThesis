Listings \ref{lst:2d-fft-code} shows an example of computing the range-velocity spectrum from a radar data cube using the two-dimensional Fast Fourier Transform method.
The function takes a single argument, which is the radar data cube.
The cube is three-dimensional with the first dimension corresponding to the receivers,
second dimension to the chirps and third dimensions to the samples of each chirp.

\begin{lstlisting}[
    language=Python,
    caption={Code example for estimating the range-velocity spectrum for a radar data cube using the 2D-FFT algorithm. (Part 1/2)},
    label={lst:2d-fft-code}
]
import numpy
def range_velocity(frame):
    K, M, N = frame.shape
    C_1 = 1; C_2 = 2  ## Stetson-Harrison method
    lower = int((M/2-C_1/2)+0.5)
    upper = int((M/2+C_1/2)+0.5)

    range_filter_coefficients = numpy.diag(
        [ 0 if c in range(0, C_2) else 1 for c in range(N) ]
    )
    velocity_filter_coefficients = numpy.diag(
        [ 0 if c in range(lower, upper) else 1 for c in range(M) ]
    )

    bins = numpy.mean(frame, axis=0)
    fast_time_fft = numpy.fft.fft(bins, axis=1)
    slow_time_fft = numpy.fft.fft(fast_time_fft, axis=0)
    shifted = numpy.fft.fftshift(slow_time_fft, axes=(0))
    
    return abs(
        velocity_filter_coefficients 
        @ (range_filter_coefficients @ shifted.T).T
    )**2
\end{lstlisting}