Listings \ref{lst:2d-music-code-1}--\ref{lst:2d-music-code-2} show an example of applying the \gls{2d-music} algorithm to a radar data cube
with dimensions $\gls{numrcv} \times \gls{numchirps} \times \gls{numsamples}$, where $\gls{numrcv}$ is the number of active receivers, 
$\gls{numchirps}$ is the number of recorded dwells and $\gls{numsamples}$ is the number of samples per dwell.

The first argument for the function defined in Listing \ref{lst:2d-music-code-2} is the radar data cube.
The second and third are respectively the range and \gls{aoa} bins the spectrum shall be estimated for.
The fourth argument is the slope of the chirp $\gls{slope} = \gls{bandwidth} \div \gls{chirptime}$, where $\gls{slope}$ is the slope of the chirp,
\gls{bandwidth} is the bandwidth of the chirp during sampling and \gls{chirptime} is the duration of the chirp.
The fifth argument is the sampling frequency and the sixth and final argument is the carrier frequency of the modulated chirp signal.

The number of targets $\gls{numtargets}$ in the algorithm is estimated using \gls{aic}, as described by Wax and Kailath \cite{wax-kailath-85}.

\begin{lstlisting}[
    language=Python,
    caption={Code example for estimating the range-azimuth spectrum for a radar data cube using the 2D-MUSIC algorithm. (Part 1/2)}\label{lst:2d-music-code-1}
]
import numpy
from scipy import constants
from scipy import pi as PI
def music(frame, ranges, angles, slope, fs, fc):
    K, M, N = frame.shape
    wlen = constants.c / fc
    d = wlen / 2
    p1 = 2; p2 = 2
    q1 = K-p1; q2 = N-p2

    cov_mtx = numpy.mean( numpy.array(
        [ covariance_FBSS(frame, q1, q2, m) for m in numpy.arange(M) ]
    ), axis=0 )
\end{lstlisting}
\newpage

Listing \ref{lst:2d-music-code-2} is continuation to listing \ref{lst:2d-music-code-1}.
\begin{lstlisting}[
    language=Python,
    caption={Code example for estimating the range-azimuth spectrum for a radar data cube using the 2D-MUSIC algorithm. (Part 2/2)}\label{lst:2d-music-code-2}
]
    eigvals, eigvecs = numpy.linalg.eigh(cov_mtx)
    # sort largest first so noise subspace is in the end
    sort_order = numpy.flip(numpy.argsort(eigvals))
    eigvecs = eigvecs[:,sort_order]
    W = eigvecs[:, L+1:]

    p = len(eigvals)
    def AIC(l):
        numerator = numpy.prod(eigvals[l:]) ** (1/(p-l))
        denominator = (1 / (p-l)) * numpy.mean(eigvals[l:])
        exponent = ((p-l)*(K))
        sum_factor = 2*l*(2*p-l)

        return -2*numpy.log10(
            (numerator/denominator)
        ) ** exponent + sum_factor

    L = numpy.argmin(numpy.array([AIC(l) for l in range(p)]))
    steering_vec = lambda theta: numpy.exp(
        1j*((2*PI)/wlen) * d*numpy.arange(q1) * numpy.sin(theta)
    )
    
    range_vec = lambda r: numpy.exp(
        1j*2*PI * ((2*r)/constants.c) * slope*numpy.arange(q2)*(1/fs)
    )
    
    alpha = lambda a, r: numpy.outer(
        steering_vec(a), range_vec(r)
    ).reshape((-1,1), order='F')
    
    range_spectrum = lambda r, theta: 1 / (
        alpha(theta, r).conj().T @ W @ W.conj().T @ alpha(theta, r)
    )
    
    P = lambda theta : numpy.vectorize(range_spectrum)(ranges, theta)
    spectrum = numpy.array([P(theta) for theta in angles])
    return abs(spectrum)**2
\end{lstlisting}