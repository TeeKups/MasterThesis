Tietokoneiden laskentatehon ja sensorien kyvykkyyden kasvu avaa uusia mahdollisuuksia aktiviteetin tunnistukselle,
jolle onkin jo esitetty useita mahdollisia sovelluksia.
Merkittävimpiä sovelluksia ovat muun muassa turvallisuus- ja terveyssovellukset,
kuten myymälävarkauksien ja ilkivallan tunnistaminen, kaatumisen tunnistaminen, sekä hengitystiheyden havainnointi.
Joitain kaupallisia sovelluksia on jo saatavilla,
kuten älykelloista ja -puhelimista löytyvät terveyssovellukset,
jotka osaavat tunnistaa joitain liikuntamuotoja.

Tyypillisesti aktiviteetintunnistukseen käytetään puettavia sensoreita.
Gyroskoopeilla ja kiihtyvyysantureilla saadaankin seurattua avainkohtien liikkeitä hyvin tarkasti,
mikä mahdollistaa suorityskykyisen aktiviteetintunnistuksen,
kunhan sensoreita on riittävästi.
Puettavien sensoreiden varjopuolena kuitenkin on,
että ne saattavat olla epämukavia ja epämuodikkaita.
Lisäksi ne vaativat suostumuksellista pukemista,
mikä saattaa rajoittaa käyttökohteita.
Puettavien sensoreiden olisi myöskin syytä olla akkukäyttöisiä,
mikä tekee niistä epäluotettavia verrattuna verkkovirtaan kytkettäviin sensoreihin.

Aktiviteetin tunnistuksessa näkyvän spektrin valon kuvaaminen (RGB kamera)
on perinteisesti ollut suosituin vaihtoehto etämittaukselle.
Valitettavasti RGB kamera on herkkä näköesteille ja valo-olosuhteille,
mikä rajoittaa sen käytettävyyttä.
Lisäksi RGB kamera on luonnostaan yksityisyyttä loukkaava.
On selvää, että muille etämittausmenetelmille on kysyntää.
Valtaosa aktiviteetin tunnistukseen saatavilla olevista dataseteistä
kattaa vain puettavia sensoreita sekä RGB videon.
Jotta etämittauksella suoritettavaa aktiviteetintunnistusta voidaan kehittää,
on luotava lisää datasettejä, jotka kattaavat muitakin etämittausmenetelmiä.

Tässä diplomityössä luotiin liikuteltava useamodaalinen nauhoitusjärjestelmä.
Järjestelmä kattaa RGB kameran, stereonäköä hyödyntävän syvyyskameran,
matalaresoluutioisen infrapunakameran, mikrofonin, sekä millimetriaaltotutkan.
Tämä diplomityö dokumentoi käytetyt sensorit,
toteutetun ohjelmiston arkkitehtuurin,
sekä toteutetun ohjelmiston tuottamat dataformaatit.
Toteutettu järjestelmä jäi hieman epävakaaksi,
mutta se soveltuu kuitenkin pienimuotoisten datasettien keräämiseen.
Epävakaus johtui luultavasti joko käytetystä tutkalaitteesta tai ohjelmointivirheestä.
Jos epävakaus saadaan korjattua, järjestelmä soveltuu myös suuremman mittakaavan datasettien keräämiseen.